\chapter{Introduction}
\label{sec:intoduction}

Imagine the following situation: Brittany, a friend of yours, is describing her new boyfriend to you on the phone. She tells you "Marc is such a nice guy. He is tall, hard-working, sometimes a bit foolish but has a good sense of humor."
After hearing this description, what height do you think Marc has? You might have something in mind like "he is fairly taller than average", but it is hard to come up with a specific number. The actual meaning of gradable adjectives like \textit{tall} seems to be not really clear. Sarah, a friend of Brittany's, might think of Marc as a 2.10m person, because she is married to a basketball player (as well as Brittany) and therefore often surrounded by people who are around 2 meters in height. Martin, a 9-year old, might consider all people as \textit{tall} who are bigger than him and therefore he could think that Marc is anything above 1.30m. Actually, Marc is 2.03m tall, which makes Sarah's guess a lot better than Martin's, because Sarah could take into account extra information, e.g. Brittany's usual environment.\\

This example tries to point out some properties of the resolution of vague utterances like the one above:
\begin{itemize}
\item Unclear semantics of \textit{tall}. It seems to be used after a certain threshold.
\item The role of comparison class. \textit{Tall} carries a different meaning when talking about humans, than when talking about mountains or buildings.
\item The role of prior expectation about height, i.e. what does one think is "normal height"?
\end{itemize}
Despite the vague meaning of gradable adjectives, people seem to use them very often in an understandable way. What mechanisms in human cognition do allow the inference of the meaning? Why is this imprecise description of properties still present in today's language and communication system?\\

To get a broader view of the topics associated with these questions we will look at the theoretical context which will provide strategies to examine these issues.
The following work is organized as follows. Chapter 2 introduces the associated scientific fields like pragmatics and evolutionary game theory in a very general way. Also some basic notions used in the work like the term "vagueness" and meaning in language use are explained, as well as the relevance of computer simulation methods. Chapter 3 introduces a mathematical framework for modeling speaker- and listener-behavior in certain communication situations. Also, an extension to the framework is given and evaluated. In chapter 4, the actual simulation is explained, containing basic assumptions and underlying principles, as well as an explanation of the actual procedure. Chapter 5 will then provide information about the gathered data and present the results, as well as possible interpretations. Section 2 can be skipped by readers who are already familiar with the described topics.